\chapter*{Conclusion générale et perspectives}
Ce rapport est le résultat du travail mené par DATAVORA dans le cadre d'un projet de fin d'études pour obtenir le diplôme National d’Ingénieur en Informatique, dans le cadre du projet de fin d'études.
\vspace{0.8cm}

Notre travail consiste à concevoir un système de Bot générique distribué. Malgré les changements qui ont eu lieu durant la période du stage, vu que la majeure partie du projet a été réalisée en télétravail (covid-19), et bien que nous avons rencontré des obstacles techniques, nous аvons pu аtteindre les objectifs que nous nous sommes fixés.
\vspace{0.8cm}

Du point de vue des connaissances acquises, ce travail a été très instructif. D'une part, cela nous donne l'opportunité d'entrer dans la vie professionnelle et d'autre part, cela nous donne l'occasion de confirmer nos connaissances en développement Python et d'entrer en contact étroit avec de nombreux aspects des systèmes des bots. 
\vspace{0.8cm}

De plus, de point de vue technique, le projet a été très riche car il nous a donné l'opportunité d'appliquer les différentes compétences que nous avons acquises au cours de nos études universitaires et d'utiliser diverses technologies (Scrapy, Django, React, etc.) tout en travaillant dans une grande entreprise avec une hiérarchie professionnelle et d'apprendre à gérer du temps et d'énergie en collaboration avec une équipe innovante et motivante. 
\vspace{0.8cm}

Ce travail a atteint son objectif, mais comme tout projet humain, il doit donc être amélioré. Dans cet esprit, le système de bot peut être amélioré surtout pour le cas de résilience. En effet, Scrapy ne fournit rien d'intègre, à part les files d'attente persistantes de Scrapyd, ce qui signifie que les travaux ayant échoués redémarreront dès que le nœud sera de retour. Nous devrons alors construire une solution de surveillance et de mise en file d'attente distribuée (basée sur Kafka ou RabbitMQ) qui redémarrera les analyses ayant échoués.
\vspace{0.8cm}

Toutefois, il est utile de noter que les données collectées par notre système de bot vont être exportées vers l'équipe data science de DATAVORA. Ces derniers vont appliquer un processus de collecte, d’organisation et d’analyse de données brutes récoltées par notre système de bot, afin de découvrir des tendances et des corrélations cachées permettant de tirer des conclusions prédictives et stratégiques pour nos clients.










\addcontentsline{toc}{chapter}{Conclusion générale}
\markboth{Conclusion générale}{}
