\chapter*{Introduction générale}
\addcontentsline{toc}{chapter}{Introduction générale} % to include the introduction to the table of content
\markboth{Introduction générale}{} %To redefine the section page head

De nos jours, le monde devient de plus en plus digital et notre environnement évolue au rythme du développement technologique. Nous sommes alors dans une course géante pour l'innovation, où tous les paramètres changent très vite. Cependant, des nouveaux outils et concepts émergent, en particulier le Big Data.

Cette nouvelle technologie peut parfaitement s'adapter à l'évolution des comportements et des attentes des consommateurs dans le cadre de l'e-commerce. En effet, ces derniers peuvent faire tous leurs achats sans quitter leurs domiciles et cela grâce au commerce électronique qui est un moyen peu coûteux et relie les intervenants pour effectuer des opérations commerciales en épargnant le temps et l'argent.

Les consommateurs génèrent ainsi un grand volume des données qui sont collectées et stockées pour que les responsables des sites e-commerce puissent les utiliser et ce afin de prendre les meilleures décisions et améliorer leurs activités.
La clé du succès du Big Data n'est pas seulement la quantité des données collectées par l'entreprise mais aussi la manière dont l'entreprise les traite et les utilise réellement.

Dans ce contexte, la solution développée par DATAVORA et ce présent, contiennent quatre chapitres organisés comme suit:
\begin{itemize}
    \item Le premier chapitre est consacré à présenter le contexte du projet à travers une présentation de l’organisme d’accueil, ainsi que la problématique, les objectifs du projet, l'état de l'art et l'étude, la critique de l'existant et la méthodologie adoptée tout au long du stage.
    \item Le deuxième chapitre détaille la solution utilisée pour l'extraction et l'indexation des données.
    \item Pour le troisième chapitre, nous mettons l'accent sur la partie de distribution de notre projet.
    \item Le dernier chapitre est consacré à la partie de Monitoring.
\end{itemize}
\newpage
Ainsi, au cours de ces chapitres, nous précisons l’architecture physique et logique adoptée ainsi que la décomposition modulaire de la solution. \\

Enfin, nous clôturons le présent rapport en concluant le travail effectué et en exposant quelques perspectives qui peuvent l'enrichir.